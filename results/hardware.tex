\subsection{Design Faults}
In this first revision of the PCB, a few design faults were made, and some small hacks had to be made to compensate.

\subsubsection{Voltage Regulator}
The first flaw encountered was the voltage regulator that was connected to ground, but through a 100$\Omega$ resistor and a capacitor.
This forced the voltage output up to 4.15V and may have bricked the FPGA that has a voltage maximum at 3.95V.
The reason for this flaw was a failure to read the full specifications in the data sheet.

\subsubsection{SD Card Connector}
After mounting the SD Card connector we found that all the pin-outs on the SD card were inverted.
The easiest solution was to the turn the SD Card connector $180^{\circ}$ and mount part of it outside the board.

\subsubsection{MCU Clock}
There was a fault in the design of the 48MHz external clock pads for the MCU. This was due to failure of wiring the pads to the right pins on the external crystal.
To fix this a crystal with a smaller surface and only two pins was bought.
This flaw could have been avoided had the PCB been printed on an A4 sheet, and the component sizes verified using real components.
Unfortunately, the components were not available at the time of ordering the PCB.

\subsubsection{Decoupling Capacitors}
We used an 0603 footprint for all capacitors and resistors.
We then had to change to a bigger 0805 or 1208 footprint when mounting a capacitor larger then 10$\mu$F.
A capacitor at 10$\mu$F does not exist in an 0603 footprint.
In the heat of the last few days, the PCB team forgot about this, and therefore the board does not have 100$\mu$F decoupling capacitors on the PCB.

\paragraph{G\_CLOCK Problems on Input}
Because of later problems with the PiCamera we had to look for other input solutions.
The HDMI input clock needs to be connected to G\_CLOCK pins on the FPGA.
Unfortunately, only TDMS 0, TMDS 1 and TMDS 2 had been connected to G\_CLOCK.
Because this wiring is impossible to change afterwards, an HDMI cable was opened and the cable's cable's internal pins were switched to make it work.
Only HDMI input requires the clock to be mapped to a G\_CLOCK pin,
thus this was not discovered during the testing that had been done before the PCB had been ordered.

\paragraph{Data Communication Between Devices}
HDMI also has an I2C interface that it uses to communicate its display data \cite{wikiddc}.
This includes supported devices, monitor parameters, etc.
This is not required for HDMI video transfer to actually work, but on most computers the HDMI will not output anything unless you send in display data or force output.
To make this job easier, routing could have been added to the I2C pins on the HDMI connector.

