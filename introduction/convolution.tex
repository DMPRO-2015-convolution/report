\section{Convolution}
\subsection{Calculus}
Mathematically, convolution is an operation that takes two input functions to create a third function:

\[
    (f * g)(t) = \int^{\infty}_{-\infty}{f(\tau)g(t-\tau) d\tau}
\]

There is also a discrete variant:

\[
    (f * g)(n) = \sum^{\infty}_{m=-\infty}{f(m)g(n-m)}
\]

A monochrome image can be thought of as a 2D matrix with a width $W$ and height $H$.
Let $I(x, y)$ be a function for accessing a specific value in the image matrix.
Let $K(x, Y)$ be a function for accessing a specific value in a special matrix called the \textit{kernel},
or $0$ if $x$ or $y$ is out of bounds for the matrix.

The convolution of the image and the kernel is then:
\[
    C(x, y) = \sum^{H}_{h=0} \sum^{W}_{w=0}{I(w, h)K(w - x, h -y)}
\]

Depending on the kernel matrix,
various effects like blurring or sharpening may be applied to the image with convolution.
For a colored RGB image convolution is performed once per color channel.

\subsection{Matrix Operation}
As described above, an image and a kernel may be convolved with pure calculus by defining functions for accessing matrices.

It is possible to take advantage of the matrix-nature of the image and kernel and conceptualize convolution in a different way.

For each pixel in the image matrix, overlay the kernel with its center on the pixel.
Multiply each pixel-kernel value that are overlayed and add these products together to get the new value for the center pixel.
This is the method that is described in the task description.

TODO: nice illustration of convolution

