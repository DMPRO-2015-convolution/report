\section{Discussion}

\subsection{Camera Input}
The backup solution presented in \ref{subsec:RiskAssessment} where input from the camera is provided through HDMI directly to the FPGA might have been a better option from the start.
Our decision to send the data through the MCU on its way to the FPGA was based on the wrong assumptions; It was rather hard to find a camera module that interfaced easily with the MCU.


\subsection{Image Compression}
TODO: Verify this; not final

The bottleneck in our system is not the image processing itself, but how to ensure enough data is available at all times.
There are several ways we could have mitigated this problem.

First of all we could have increased the dimensions, that is, increase the width of our buses. Second, we could have tried to increase the clock rate. This is not easy to do on the EFM, but the processor implementation can always be pipelined more until we reach the limit of what the FPGA can manage.

A third, less obvious option, is to compress the data. We could have employed both lossy and lossless encodings to reduce the size of the data and thus increase the effective throughput. The indexed colours can be viewed as a form of lossy compression.
More advanced methods can be applied, but these would obviously require more work.
In addition, there is no guarantee the MCU can do more advanced compression at the rate required.

\subsection{EBI and SRAM Throughput}
As stated in section \ref{subsec:EbiThroughput}, we achieved a throughput of \unit[384]{Mbps}.
This must be considered to be an absolute maximum as there is no guarantee the MCU is capable of delivering this amount of data anyway.

When it comes to SRAM, the throughput measured in section \ref{subsec:SramThroughput} is obviously limited by the achievable write speed.
Even though the chip has an access time of \unit[10]{ns}, writing requires a write pulse width of minimum \unit[8]{ns}, and since we have a duty cycle of \unit[50]{\%}, the lower bound on a write is \unit[16]{ns}.
Thus the specifications for the SRAM chips guarantees a write speed of $(2\cdot\unit[8]{ns})^{-1} \cdot \unit[16]{bits} / 2 = \unit[500]{Mbps}$.

Our speed being lower may be due to latencies in the FPGA as the signals have to propagate from the registers, through logic and some multiplexers before reaching the bus and finally the SRAM chips them selves.
