\subsection{EBI and SRAM Throughput}
As stated in Section \ref{subsec:EbiThroughput}, a throughput of \unit[384]{Mbps} was achieved over EBI.
This must be considered to be an absolute maximum as there is no guarantee the MCU is capable of delivering this amount of data anyway, but since the final design did not utilize EBI for video transfers, the throughput over EBI is not a limiting factor.

When it comes to SRAM, the throughput measured in Section \ref{subsec:SramThroughput} is limited by the achievable write speed.
Even though the chip has an access time of \unit[10]{ns}, writing requires a write pulse width of minimum \unit[8]{ns}, and since the computer uses a duty cycle of \unit[50]{\%}, the real lower bound on a write is \unit[16]{ns}.
Thus the specifications for the SRAM chips guarantees a write speed of $(2\cdot\unit[8]{ns})^{-1} \cdot \unit[16]{bits} / 2 = \unit[500]{Mbps}$ using the implemented write cycle.

The speed being lower may be due to latencies in the FPGA as the signals have to propagate from the registers, through logic and some multiplexers before reaching the bus and finally the SRAM chips themselves. In addition, more experimenting with the write and read cycles could probably have improved the clock rate slightly giving a slightly higher throughput.
