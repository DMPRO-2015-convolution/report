\section{High-Definition Multimedia Interface (HDMI)}
Our implementation of HDMI on the FPGA was based on the Xilinx application note 495\cite{xapp495} and its reference design files.

We used an HDMI input as the source for the picture to be convoluted, and an HDMI output to show the result.
HDMI uses TMDS at the physical layer, and our implementation use the native TMDS I/O interface that is featured by the Spartan-6 FPGAs.

\subsubsection{HDMI input}
To get pixel information from the HDMI input, we have a verilog module called dvi\_decoder. This module sends the TMDS clock signals to a differential buffer, a Spartan 6 primitive called BUFIO2 and then into a PLL to get the required clocks. The required clocks for making HDMI signals into 10 bit words are:
\begin{itemize}
    \item   Pixel clock, which is the same rate as the TMDS clock.
    \item   2x pixel clock, used when making 5 bit words into 10 bit words.
    \item   10x pixel clock, used as IO clock.
\end{itemize}

After the clock is retrieved from the TMDS signals, it sends the TMDS signals for each of the colors into their own instance of the decode submodule.


The output data is then serialized, using the serdes\_n\_to\_1 module.
These serial bitstreams are then sent to the OBUFDS cores to produce the wanted output differential signals.
The clock is produced using the PLL\_BASE module.
The wanted pixel clock is then sent to the ODDR2 module, to get the wanted differential signal output.
\subsubsection{HDMI output}
