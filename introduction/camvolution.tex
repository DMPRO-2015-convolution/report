\section{Camvolution}
The given task can be solved in multiple ways. This section presents our interpretation and the requirements for our system.

\subsection{Area of Focus}
As the task places no restrictions on what we should focus on as long as the computer performs convolution, we considered several possibilities, some inspired by the previous project reports:

\begin{description}
    \item[Energy efficiency] Do convolution in an energy efficient manner.
    \item[Generality] Make it possible to use the system for a wide variety of tasks.
    \item[Performance] Maximize the throughput of the system.
\end{description}

In the end, the group decided to focus on performance as this would give us a more engaging demonstration, possibly with convolution performed on a live video stream.

\subsection{Requirements}
A set of goals or requirements is useful to guide the efforts in the direction of the assignment, and to ensure a demo can be held at the end of the project.

\subsubsection{Functional Requirements}
The functional requirements set for this project is listed in table \ref{tab:FunctionalRequirements} and a more thorough descriptions follows.

\begin{table}[h]
    \centering
    \begin{tabular}{lp{12cm}l}
        Name & Description & Priority \\
        \hline
        FR1 &
            The processor should do convolution on a 2 dimensional image with at least a 3x3 convolution kernel &
            High \\
        FR2 &
            The processor should have a video port to show graphical output &
            High \\
        FR3 &
            It should be able to use a camera as input &
            Medium \\
        FR4 &
            The machine should boot and operate without an external computer &
            Medium
    \end{tabular}
    \caption{Functional Requirements}
    \label{tab:FunctionalRequirements}
\end{table}

\paragraph{FR1}
The processor should be able to do convolution.
As the assignment states, this should be done on a 2 dimensional structure for which we have chosen an image.
Also, convolution is meaningless with a convolution kernel smaller than 3x3, thus this is also included in the requirements.

\paragraph{FR2}
It follows directly from the \nameref{subsec:assignment-text} that we must be able to display output from an application that produces graphical data.
To accomplish this, the computer is required to have a video port that can output graphics.

\paragraph{FR3}
As we decided to focus on performance, one of our goals is to be able to do convolution on a live video stream.
This requires an input stream which we have decided should come from a video camera connected to our computer.

\paragraph{FR4}
To make the demonstration a bit smoother and ensure our computer can accomplish our goals independently, we also require that the machine should be able to boot and operate without an external computer.


\subsubsection{Non-functional Requirements}

\begin{table}[h]
    \centering
    \begin{tabular}{lp{12cm}l}
        Name & Description \\
        \hline
        NFR1 &
            The processor should be implemented on a Xilinx FPGA \\
        NFR2 &
            The processor should use Silicon Labs EFM32 microcontroller as I/O processor \\
        NFR3 &
            The cost of developing the system should not exceed a budget of 10 000 NOK \\
    \end{tabular}
    \caption{Non-functional Requirements}
    \label{tab:NonFunctionalRequirements}
\end{table}

In addition to our functional requirements, the task at hand also enforces requirements not related to the functionality of the computer produced. These are listed in table \ref{tab:NonFunctionalRequirements}.
i
