\section{Testing}

In this section the tests performed to verify the functionality of the different components that makes up the system, and also some of the connections between them, is presented.
Some tests were performed on the physical hardware, while others were pure software simulations.

\subsection{FPGA}
The SRAM and EBI managers were tested using VHDL test benches before being implemented on the FPGA.

For the processor core, Chisel test benches were used, as testing is quite well supported by the language itself.
The processor core was tested using VHDL test benches near the end of the procjet, but the tests failed with invalid values being reported by the simulation tool.

\subsubsection{EBI Throughput} \label{subsec:EbiThroughput}
During the project, the throughput over EBI between a EFM MCU and a Spartan-6 FPGA was tested by sending a sequence of increasing numbers which were verified on the FPGA.
A set-up time of one cycle was required in addition to the mandatory cycle always introduced by the MCU EBI controller for data to be transferred correctly.

Since the MCU is running with a clock speed of 48MHz, this resulted in a maximum throughput of $\unit[16]{bits} \cdot \unit[48]{MHz} / 2 = \unit[384]{Mbps}$.
From Table \ref{tab:BitRates}, it can be concluded that this is sufficient to accommodate all options listed except the most data intensive.

During the testing, a default FIFO queue implementation from Xilinx was used to cross the clock domain.
As EBI is asynchronous and therefore does not provide a clock, the last value sent was not received on the FPGA.
As a workaround, a final \textit{don't care} value was sent over the bus to push the final data value through the FIFO at the FPGA.

\subsubsection{SRAM Throughput} \label{subsec:SramThroughput}
During SRAM testing, several clock rates were tried to determine the speed achievable to SRAM.
After some binary search through the frequency spectrum, $\unit[49]{MHz}$ was found to be the highest working frequency.
This gives a throughput of $\unit[49]{MHz}\cdot\unit[16]{bit} = \unit[784]{Mbps}$ for reading.
During writing however, this number must be halved as a write requires two cycles (one for address set-up and one for writing), giving a throughput for writing of $\unit[392]{Mbps}$.

\subsection{Hardware}
After soldering the board, each component was singled out and tested to ensure proper functionality.

\subsubsection{SRAM Connection}
The resistance between each SRAM pin and the corresponding header pin was measured to ensure all wires were connected properly.
It was discovered that two of the address pins for the second memory chip had been exchanged, but it was concluded this must be due to a swapping of the labels on the board, not the actual wires.
Nevertheless, this would not make a difference even if the wires were swapped as the data would be swapped twice, first when writing, then when reading, resulting in the correct data being read.

The chips were confirmed to be working by configuring the FPGA with an image that performed the following steps:
\begin{enumerate}
    \item Iterate through all memory addresses and write the lower 16 bits of the address XOR some random values (in our case, 0xDEAD and 0xBEEF for SRAM1 and SRAM2 respectively).
    \item Start over at the lowest address, read the data and compare with the expected value.
    \item Light the LED with a light characteristic dependent on the result:
        \begin{itemize}
            \item Fixed on success.
            \item Occulting on SRAM1 failure.
            \item Flashing on SRAM2 failure (if SRAM1 was successfully read).
        \end{itemize}
\end{enumerate}

After fixing some malfunctioning solder points not detected during initial measuring, the test passed.

\subsubsection{Flashing the MCU}
The ARM Debug header on the computer was connected to the corresponding header on an EFM32GG development kit. The MCU was successfully flashed, and verified by inspecting variables in the debug mode of Simplicity Studio.

\subsubsection{SD Card}
Reading and writing to the MicroSD card through the MCU was tested by reading a small file from the card into a buffer and inspecting it with the debug mode of Simplicity Studio. It turned out the wires for the connector had been flipped, but the test passed after rotating the SD card slot $180^{\circ}$ to make the connections align.

\subsubsection{Configuring the FPGA}
The FPGA was configured both using JTAG and through the MCU with a bit file. The file contained a configuration for the FPGA that would allow the on-board LED to be controlled using one of the buttons. The button controlled the LED after configuration, thus it was concluded that the configuration was successful.

\subsubsection{HDMI Output}
HDMI output was confirmed to be working by configuring the FPGA with bit files outputting an image generated on the fly.
Several images were tried, some filled with a single colour, and some with a repeating gradient pattern.

The image showed up on a connected display, thus it was concluded that the HDMI output was working.

\subsection{HDMI Input}
When trying to configure the FPGA for receiving HDMI through the second port it was discovered that the pins used for the incoming clock signal was connected to a non-clock input pin.
Ironically, all other pins used for the given HDMI connector were clock pins, and a regular HDMI cable was opened and one of the colour wire pairs exchanged with the clock pair.

HDMI input was confirmed to be working by forwarding the HDMI input to the HDMI output.
