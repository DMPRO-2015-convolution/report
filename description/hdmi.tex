\section{High-Definition Multimedia Interface (HDMI)}
Our implementation of HDMI on the FPGA was based on the Xilinx application note 495 \cite{xapp495} and its reference design files.

We used an HDMI input as the source for the picture to be convolved, and an HDMI output to show the result.
HDMI uses TMDS at the physical layer, and our implementation uses the native TMDS I/O interface that is featured by the Spartan-6 FPGAs.

\subsection{HDMI Input}
The TMDS clock signals are sent to a differential buffer, a Spartan-6 primitive called BUFIO2, and then into a PLL to get the required clocks. The required clocks for making HDMI signals into 10 bit words are:
\begin{itemize}
    \item   Pixel clock, which is the same rate as the TMDS clock.
    \item   2x pixel clock, used when making 5 bit words into 10 bit words.
    \item   10x pixel clock, used as IO clock.
\end{itemize}

After the clock is retrieved from the TMDS signals, it sends the TMDS signals for each of the colours into their own instance of the decode submodule, together with the PLL generated pixel clock that is based on the incoming TMDS signal clock. After the colour channel signals are decoded, we have the needed colour information about the pixel. This is written to a FIFO queue which is to be read by the convolution processor.

\subsection{HDMI Output}

The processed pixel is read from a FIFO queue. It is a 24-bit word, with each colour being 8 bit. A pixel clock for the transmitting end of the HDMI interface is generated by a PLL. Each colour and the pixel clock is serialized with a frequency of 10x the pixel clock before being sent out as differential signals.
