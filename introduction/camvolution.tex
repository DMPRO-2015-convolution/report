\section{Camvolution}
The given task\cite{assignment-text} is very open-ended and can be solved in multiple ways.
This section presents the groups' interpretation and the requirements for the system.

\subsection{Area of Focus}
As the task places no restrictions on what should be the area of focus on as long as the computer performs convolution, several possibilities were considered, some inspired by the previous project reports:

\begin{description}
    \item[Energy Efficiency] Do convolution in an energy efficient manner.
    \item[Generality] Make it possible to use the system for a wide variety of tasks.
    \item[Performance] Maximize the throughput of the system.
\end{description}

After some deliberation,
the group decided to focus on performance.
We decided to perform convolution on a video stream and output this manipulated video.
This video stream should come from a video camera attached to the computer.

\subsection{Requirements}
A set of goals or requirements is useful to guide the efforts in the direction of the assignment, and to ensure a demonstration can be held at the end of the project.

\subsubsection{Functional Requirements}
The functional requirements set for this project is listed in Table \ref{tab:FunctionalRequirements} and more thorough descriptions follow.

\begin{table}[h]
    \centering
    \begin{tabular}{lp{12cm}l}
        Name & Description & Priority \\
        \hline
        FR1 &
            The processor should do convolution on a two-dimensional image with at least a 3x3 convolution kernel &
            High \\
        FR2 &
            The system should be able to display output on an external screen &
            High \\
        FR3 &
            The system should be able to use a camera as input &
            Medium \\
        FR4 &
            The system should boot and operate without an external computer &
            Medium
    \end{tabular}
    \caption{Functional Requirements}
    \label{tab:FunctionalRequirements}
\end{table}

\paragraph{FR1}
The processor should be able to do convolution.
As the assignment text states, this should be done on a two-dimensional structure, for which an image was chosen.
Since image convolution is meaningless with a convolution kernel smaller than 3x3,
this is also included in the requirements.

\paragraph{FR2}
It follows directly from the assignment text that the system must be able to display output from an application that produces graphical data.
To accomplish this, the computer is required to have a display port that can output graphics.

\paragraph{FR3}
As it was decided to focus on performance, one of the goals set was to be able to do convolution on a live video stream.
This requires an input stream which it was decided should come from a video camera connected to the computer.

\paragraph{FR4}
To make the demonstration a bit smoother and ensure the computer can accomplish the goals independently, it was also also required that the system should be able to boot and operate without an external computer.

\subsubsection{Non-Functional Requirements}

In addition to the functional requirements, the task at hand also enforces requirements not related to the functionality of the computer produced. These are listed in Table \ref{tab:NonFunctionalRequirements}.

\begin{table}[h]
    \centering
    \begin{tabular}{lp{12cm}l}
        Name & Description \\
        \hline
        NFR1 &
            The processor should be implemented on a Xilinx FPGA \\
        NFR2 &
            The computer should use a Silicon Labs EFM32 microcontroller as an I/O processor \\
        NFR3 &
            The cost of developing the system should not exceed a budget of 10.000 NOK \\
    \end{tabular}
    \caption{Non-Functional Requirements}
    \label{tab:NonFunctionalRequirements}
\end{table}

