\subsection{Camera Input}
\label{sec:camera_discussion}
In the early stages of the project,
we looked at various camera modules online and tried to assess how they would work in our system.
Some modules were very simple, and some had many features.
The PiCamera was the most feature-complete module we looked at.
We looked into the camera control protocol for the PiCamera,
and found \textit{some} information about it,
like the open source C and Python camera control libraries for the Raspberry Pi
and the pinouts for the ribbon connector.
From what we found, we interpreted that it shouldn't be too hard to implement camera control with the MCU or FPGA.

When the time came to make a choice for the camera module and order it,
the PiCamera was the only option that was available from the retailers we were using,
so the choice was easy.

Had we looked for more information about the PiCamera online before this,
we would have perhaps found the Raspberry Pi engineer recommending not using the PiCamera for an FPGA project\footnote{\url{https://www.raspberrypi.org/forums/viewtopic.php?t=119977&p=812191}} or the unclaimed \$1000 bounty for an open-source driver for the PiCamera\footnote{\url{https://parallella.org/forums/viewtopic.php?f=10&t=2514}}.

The backup solution presented in \ref{subsec:RiskAssessment} where input from the camera is provided through HDMI directly to the FPGA might have been a better option from the start.
Our decision to send the data through the MCU on its way to the FPGA was based on the wrong assumptions;
It was rather hard to find a camera module that interfaced easily with the MCU.


