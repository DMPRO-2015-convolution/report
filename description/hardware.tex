\section{PCB Design}
This section describes the PCB design and explains the decisions related to this.

See Table \ref{listofcomponents} for a list of components used on the board.

\subsection{Board Features}
The board has been designed to put important components close to each other,
to reduce the distance and time to transfer data between components.
Header pins are present for most components for debugging and in case something should be wrongly wired inside the board.
Figure \ref{fig:BoardLayout} shows an overview of the board with various components labelled.

\begin{itemize}
    \item Xilinx Spartan-6 FPGA
    \item Silicon Labs EFM32GG MCU
    \item Connectors
        \begin{itemize}
            \item HDMI I/O
            \item FPC Camera
            \item MicroSD
        \end{itemize}
    \item 120MHz Oscillator for FPGA
    \item 48MHz Crystal for MCU
    \item Digital I/O
        \begin{itemize}
            \item 7 Mechanical Buttons
            \item 7 Expansion Headers
            \item 2 LEDs
        \end{itemize}
    \item External memory
        \begin{itemize}
            \item 2 AS7C38098A SRAM
        \end{itemize}
\end{itemize}

\begin{figure}[h]
    \includegraphics[width=\linewidth]{img/"Overview of Camvolution kit".jpg}
    \caption{Overview of Camvolution board}
    \label{fig:BoardLayout}
\end{figure}

\subsection{Power Supply}
The board is powered by a 5V MiniUSB connector.
The input voltage is regulated with a 3.3V low-dropout (LDO) regulator, which gives power to most of the board.
This 3.3V is also connected to a 1.2V LDO regulator which powers the minimum required power pins on the FPGA.
There is a secondary power connection for backup in case the first should fail somehow.

\subsection{Clock Generation}
The FPGA oscillator is connected to an enable signal on the MCU.
To enable the 120MHz oscillator, the signal must be set to output and driven high.
This enables the MCU to start and stop the FPGA and reduce the power consumption.
The oscillator enable pin is also connected to an external input and can be set high by connecting this pin to VCC.
This can be done by adding a jumper to on the FPGA\_OSC\_EN header.

In addition to the oscillator, the MCU has one internal and one external clock.

\subsection{Connectors}
\subsubsection{HDMI}
There are two HDMI connectors on the board.
Both of them can be connected to 5V with jumpers.
This is not required for the HDMI to work, but is added in case power should be required by other devices.
The port labelled HDMI 2 is connected to header pins.
See Table \ref{HDMI_Connector} for pin-outs.

\subsubsection{FPC Camera Connector}
There is one flexible printed circuit (FPC) connector on the board intended for the Raspberry Pi Camera.
It is also connected to header pins.
See Figure \ref{fig:FpcHeader} for pin configuration.

\subsection{Digital I/O}
\subsubsection{Buttons}
The board contains seven buttons.
When a button is pushed down, the input pin will be grounded.

The two buttons are for FPGA control, and four buttons for MCU control.
The last button is connected to reset on the MCU.
See Table \ref{tab:Buttons} for info.

\subsubsection{Header Output}
The MCU has spare I/O pins that all are available on header pins.
These can be used as GPIO for the system if neccessary.
See Table \ref{tab:HeaderOut} for location of pins.

\subsection{SRAM}
\label{subsec:sram}
The FPGA is connected to two SRAM chips, one on bank 1 and another on bank 2.
The chips support a data width of 16 bits and have an access time of \unit[10]{ns} \cite{sramdatasheet}, which gives a maximum throughput of $(\unit[10]{ns})^{-1} = \unit[1,6]{Gbps}$ to each chip.
This is more than sufficient for this projects' application as can be seen in Table \ref{tab:BitRates}.

To make debugging easier, all wires connecting the FPGA and SRAM chips have been placed on headers.

\subsection{JTAG}
Both the FPGA and the MCU is connected to a JTAG header.
The MCU is connected to a 20 pin JTAG header that can be used with the generic programming header from the MCU development kit.
The FPGA is connected to a 8 pin JTAG header.
This can be routed to a debugger used for FPGA programming.
See Table \ref{tab:EfmProgrammer} for MCU JTAG header and Figure \ref{fig:FpgaProgrammer} for FPGA JTAG header.

\subsection{Programming FPGA Using MCU SPI}
The FPGA can also be programmed using serial peripheral interface (SPI) from the MCU.
INIT\_B shares the same pin as the SRAM2 data line, but when programming the FPGA the SRAM will not be active.
See Table \ref{tab:SpiProgrammer}.

\subsection{LED}
There are two LEDs on the board.
One is connected to power on the board and will light up if power is connected.
The second is programmable from the FPGA.

\subsection{External Bus Interface (EBI)}
The FPGA and the MCU is connected with a high speed EBI bus with 20 addressable pins and 16 data lines.
They are not connected to header pins.
If any wire design error had occured, there would not be any purpose of having them on headers, as it could be manually changed in the FPGA.
See Appendix \ref{app:pinouts}, Figure \ref{fig:EbiBus} and \ref{fig:EbiControl} for EBI pinouts.

